\hypertarget{vec3_8c}{
\section{/ich/Programmieren/KazadeEngine/NeHeKazmath/kazmath/src/vec3.c File Reference}
\label{vec3_8c}\index{/ich/Programmieren/KazadeEngine/NeHeKazmath/kazmath/src/vec3.c@{/ich/Programmieren/KazadeEngine/NeHeKazmath/kazmath/src/vec3.c}}
}
{\tt \#include $<$assert.h$>$}\par
{\tt \#include $<$memory.h$>$}\par
{\tt \#include \char`\"{}utility.h\char`\"{}}\par
{\tt \#include \char`\"{}vec4.h\char`\"{}}\par
{\tt \#include \char`\"{}mat4.h\char`\"{}}\par
{\tt \#include \char`\"{}vec3.h\char`\"{}}\par
\subsection*{Functions}
\begin{CompactItemize}
\item 
\hyperlink{structkm_vec3}{kmVec3} $\ast$ \hyperlink{vec3_8c_327a16c25be8c921f5c8710e14c4fdcd}{kmVec3Fill} (\hyperlink{structkm_vec3}{kmVec3} $\ast$pOut, kmScalar x, kmScalar y, kmScalar z)
\item 
kmScalar \hyperlink{vec3_8c_8e019f2b3549e0dc7b9847868ba548fa}{kmVec3Length} (const \hyperlink{structkm_vec3}{kmVec3} $\ast$pIn)
\item 
kmScalar \hyperlink{vec3_8c_aabb0b5f2fc888052679d401852e63a9}{kmVec3LengthSq} (const \hyperlink{structkm_vec3}{kmVec3} $\ast$pIn)
\item 
\hyperlink{structkm_vec3}{kmVec3} $\ast$ \hyperlink{vec3_8c_61a55fc5e4ce21322cb8141653a1d0ff}{kmVec3Normalize} (\hyperlink{structkm_vec3}{kmVec3} $\ast$pOut, const \hyperlink{structkm_vec3}{kmVec3} $\ast$pIn)
\item 
\hyperlink{structkm_vec3}{kmVec3} $\ast$ \hyperlink{vec3_8c_2b3ed559520bc2071b05f40b29f343e6}{kmVec3Cross} (\hyperlink{structkm_vec3}{kmVec3} $\ast$pOut, const \hyperlink{structkm_vec3}{kmVec3} $\ast$pV1, const \hyperlink{structkm_vec3}{kmVec3} $\ast$pV2)
\item 
kmScalar \hyperlink{vec3_8c_7ed6bea8c65144ef31156825aea0b3e5}{kmVec3Dot} (const \hyperlink{structkm_vec3}{kmVec3} $\ast$pV1, const \hyperlink{structkm_vec3}{kmVec3} $\ast$pV2)
\item 
\hyperlink{structkm_vec3}{kmVec3} $\ast$ \hyperlink{vec3_8c_eec93a828519b0ecdd20942463598c5e}{kmVec3Add} (\hyperlink{structkm_vec3}{kmVec3} $\ast$pOut, const \hyperlink{structkm_vec3}{kmVec3} $\ast$pV1, const \hyperlink{structkm_vec3}{kmVec3} $\ast$pV2)
\item 
\hyperlink{structkm_vec3}{kmVec3} $\ast$ \hyperlink{vec3_8c_07cb9ab6d64f4a458191f80eb3405324}{kmVec3Subtract} (\hyperlink{structkm_vec3}{kmVec3} $\ast$pOut, const \hyperlink{structkm_vec3}{kmVec3} $\ast$pV1, const \hyperlink{structkm_vec3}{kmVec3} $\ast$pV2)
\item 
\hyperlink{structkm_vec3}{kmVec3} $\ast$ \hyperlink{vec3_8c_e0bf920655ecb4211d501af13acaa55b}{kmVec3Transform} (\hyperlink{structkm_vec3}{kmVec3} $\ast$pOut, const \hyperlink{structkm_vec3}{kmVec3} $\ast$pV, const \hyperlink{structkm_mat4}{kmMat4} $\ast$pM)
\item 
\hyperlink{structkm_vec3}{kmVec3} $\ast$ \hyperlink{vec3_8c_7be4f0a32d26fc4e7f293febd6322c78}{kmVec3InverseTransform} (\hyperlink{structkm_vec3}{kmVec3} $\ast$pOut, const \hyperlink{structkm_vec3}{kmVec3} $\ast$pVect, const \hyperlink{structkm_mat4}{kmMat4} $\ast$pM)
\item 
\hyperlink{structkm_vec3}{kmVec3} $\ast$ \hyperlink{vec3_8c_8d949199cbd41558bac73f8f2f43ab5b}{kmVec3InverseTransformNormal} (\hyperlink{structkm_vec3}{kmVec3} $\ast$pOut, const \hyperlink{structkm_vec3}{kmVec3} $\ast$pVect, const \hyperlink{structkm_mat4}{kmMat4} $\ast$pM)
\item 
\hyperlink{structkm_vec3}{kmVec3} $\ast$ \hyperlink{vec3_8c_43d9ed3e0b77815f8e21a62d70310d43}{kmVec3TransformCoord} (\hyperlink{structkm_vec3}{kmVec3} $\ast$pOut, const \hyperlink{structkm_vec3}{kmVec3} $\ast$pV, const \hyperlink{structkm_mat4}{kmMat4} $\ast$pM)
\item 
\hyperlink{structkm_vec3}{kmVec3} $\ast$ \hyperlink{vec3_8c_7bcd2c7c5d14ca2dcea701e52450b9d9}{kmVec3TransformNormal} (\hyperlink{structkm_vec3}{kmVec3} $\ast$pOut, const \hyperlink{structkm_vec3}{kmVec3} $\ast$pV, const \hyperlink{structkm_mat4}{kmMat4} $\ast$pM)
\item 
\hyperlink{structkm_vec3}{kmVec3} $\ast$ \hyperlink{vec3_8c_340a910a28c1b7bec6c55fb7072e331f}{kmVec3Scale} (\hyperlink{structkm_vec3}{kmVec3} $\ast$pOut, const \hyperlink{structkm_vec3}{kmVec3} $\ast$pIn, const kmScalar s)
\item 
int \hyperlink{vec3_8c_050b057e12c51103781d5dec80153114}{kmVec3AreEqual} (const \hyperlink{structkm_vec3}{kmVec3} $\ast$p1, const \hyperlink{structkm_vec3}{kmVec3} $\ast$p2)
\item 
\hyperlink{structkm_vec3}{kmVec3} $\ast$ \hyperlink{vec3_8c_b37e84980fd909eeb65ff21268a9e5ac}{kmVec3Assign} (\hyperlink{structkm_vec3}{kmVec3} $\ast$pOut, const \hyperlink{structkm_vec3}{kmVec3} $\ast$pIn)
\item 
\hyperlink{structkm_vec3}{kmVec3} $\ast$ \hyperlink{vec3_8c_b1384be1801802bdd4390481ff018400}{kmVec3Zero} (\hyperlink{structkm_vec3}{kmVec3} $\ast$pOut)
\end{CompactItemize}


\subsection{Detailed Description}


Definition in file \hyperlink{vec3_8c-source}{vec3.c}.

\subsection{Function Documentation}
\hypertarget{vec3_8c_eec93a828519b0ecdd20942463598c5e}{
\index{vec3.c@{vec3.c}!kmVec3Add@{kmVec3Add}}
\index{kmVec3Add@{kmVec3Add}!vec3.c@{vec3.c}}
\subsubsection[kmVec3Add]{\setlength{\rightskip}{0pt plus 5cm}{\bf kmVec3}$\ast$ kmVec3Add ({\bf kmVec3} $\ast$ {\em pOut}, \/  const {\bf kmVec3} $\ast$ {\em pV1}, \/  const {\bf kmVec3} $\ast$ {\em pV2})}}
\label{vec3_8c_eec93a828519b0ecdd20942463598c5e}


Adds 2 vectors and returns the result. The resulting vector is stored in pOut. 

Definition at line 121 of file vec3.c.

References kmVec3::x, kmVec3::y, and kmVec3::z.

Referenced by kmQuaternionMultiplyVec3().\hypertarget{vec3_8c_050b057e12c51103781d5dec80153114}{
\index{vec3.c@{vec3.c}!kmVec3AreEqual@{kmVec3AreEqual}}
\index{kmVec3AreEqual@{kmVec3AreEqual}!vec3.c@{vec3.c}}
\subsubsection[kmVec3AreEqual]{\setlength{\rightskip}{0pt plus 5cm}int kmVec3AreEqual (const {\bf kmVec3} $\ast$ {\em p1}, \/  const {\bf kmVec3} $\ast$ {\em p2})}}
\label{vec3_8c_050b057e12c51103781d5dec80153114}


Returns KM\_\-TRUE if the 2 vectors are approximately equal 

Definition at line 268 of file vec3.c.

References kmEpsilon, kmVec3::x, kmVec3::y, and kmVec3::z.\hypertarget{vec3_8c_b37e84980fd909eeb65ff21268a9e5ac}{
\index{vec3.c@{vec3.c}!kmVec3Assign@{kmVec3Assign}}
\index{kmVec3Assign@{kmVec3Assign}!vec3.c@{vec3.c}}
\subsubsection[kmVec3Assign]{\setlength{\rightskip}{0pt plus 5cm}{\bf kmVec3}$\ast$ kmVec3Assign ({\bf kmVec3} $\ast$ {\em pOut}, \/  const {\bf kmVec3} $\ast$ {\em pIn})}}
\label{vec3_8c_b37e84980fd909eeb65ff21268a9e5ac}


Assigns pIn to pOut. Returns pOut. If pIn and pOut are the same then nothing happens but pOut is still returned 

Definition at line 283 of file vec3.c.

References kmVec3::x, kmVec3::y, and kmVec3::z.

Referenced by kmMat4LookAt(), and kmQuaternionRotationBetweenVec3().\hypertarget{vec3_8c_2b3ed559520bc2071b05f40b29f343e6}{
\index{vec3.c@{vec3.c}!kmVec3Cross@{kmVec3Cross}}
\index{kmVec3Cross@{kmVec3Cross}!vec3.c@{vec3.c}}
\subsubsection[kmVec3Cross]{\setlength{\rightskip}{0pt plus 5cm}{\bf kmVec3}$\ast$ kmVec3Cross ({\bf kmVec3} $\ast$ {\em pOut}, \/  const {\bf kmVec3} $\ast$ {\em pV1}, \/  const {\bf kmVec3} $\ast$ {\em pV2})}}
\label{vec3_8c_2b3ed559520bc2071b05f40b29f343e6}


Returns a vector perpendicular to 2 other vectors. The result is stored in pOut. 

Definition at line 91 of file vec3.c.

References kmVec3::x, kmVec3::y, and kmVec3::z.

Referenced by kmMat4LookAt(), kmPlaneFromPoints(), kmQuaternionMultiplyVec3(), and kmQuaternionRotationBetweenVec3().\hypertarget{vec3_8c_7ed6bea8c65144ef31156825aea0b3e5}{
\index{vec3.c@{vec3.c}!kmVec3Dot@{kmVec3Dot}}
\index{kmVec3Dot@{kmVec3Dot}!vec3.c@{vec3.c}}
\subsubsection[kmVec3Dot]{\setlength{\rightskip}{0pt plus 5cm}kmScalar kmVec3Dot (const {\bf kmVec3} $\ast$ {\em pV1}, \/  const {\bf kmVec3} $\ast$ {\em pV2})}}
\label{vec3_8c_7ed6bea8c65144ef31156825aea0b3e5}


Returns the cosine of the angle between 2 vectors 

Definition at line 110 of file vec3.c.

References kmVec3::x, kmVec3::y, and kmVec3::z.

Referenced by kmPlaneFromPointNormal(), kmPlaneFromPoints(), and kmQuaternionRotationBetweenVec3().\hypertarget{vec3_8c_327a16c25be8c921f5c8710e14c4fdcd}{
\index{vec3.c@{vec3.c}!kmVec3Fill@{kmVec3Fill}}
\index{kmVec3Fill@{kmVec3Fill}!vec3.c@{vec3.c}}
\subsubsection[kmVec3Fill]{\setlength{\rightskip}{0pt plus 5cm}{\bf kmVec3}$\ast$ kmVec3Fill ({\bf kmVec3} $\ast$ {\em pOut}, \/  kmScalar {\em x}, \/  kmScalar {\em y}, \/  kmScalar {\em z})}}
\label{vec3_8c_327a16c25be8c921f5c8710e14c4fdcd}


Fill a \hyperlink{structkm_vec3}{kmVec3} structure using 3 floating point values The result is store in pOut, returns pOut 

Definition at line 42 of file vec3.c.

References kmVec3::x, kmVec3::y, and kmVec3::z.

Referenced by kmGLRotatef(), kazmathxx::Vec3::operator$\ast$(), kazmathxx::Vec3::operator+(), kazmathxx::Vec3::operator-(), and kazmathxx::Vec3::operator/().\hypertarget{vec3_8c_7be4f0a32d26fc4e7f293febd6322c78}{
\index{vec3.c@{vec3.c}!kmVec3InverseTransform@{kmVec3InverseTransform}}
\index{kmVec3InverseTransform@{kmVec3InverseTransform}!vec3.c@{vec3.c}}
\subsubsection[kmVec3InverseTransform]{\setlength{\rightskip}{0pt plus 5cm}{\bf kmVec3}$\ast$ kmVec3InverseTransform ({\bf kmVec3} $\ast$ {\em pOut}, \/  const {\bf kmVec3} $\ast$ {\em pVect}, \/  const {\bf kmMat4} $\ast$ {\em pM})}}
\label{vec3_8c_7be4f0a32d26fc4e7f293febd6322c78}




Definition at line 180 of file vec3.c.

References kmMat4::mat, kmVec3::x, kmVec3::y, and kmVec3::z.\hypertarget{vec3_8c_8d949199cbd41558bac73f8f2f43ab5b}{
\index{vec3.c@{vec3.c}!kmVec3InverseTransformNormal@{kmVec3InverseTransformNormal}}
\index{kmVec3InverseTransformNormal@{kmVec3InverseTransformNormal}!vec3.c@{vec3.c}}
\subsubsection[kmVec3InverseTransformNormal]{\setlength{\rightskip}{0pt plus 5cm}{\bf kmVec3}$\ast$ kmVec3InverseTransformNormal ({\bf kmVec3} $\ast$ {\em pOut}, \/  const {\bf kmVec3} $\ast$ {\em pVect}, \/  const {\bf kmMat4} $\ast$ {\em pM})}}
\label{vec3_8c_8d949199cbd41558bac73f8f2f43ab5b}




Definition at line 199 of file vec3.c.

References kmMat4::mat, kmVec3::x, kmVec3::y, and kmVec3::z.\hypertarget{vec3_8c_8e019f2b3549e0dc7b9847868ba548fa}{
\index{vec3.c@{vec3.c}!kmVec3Length@{kmVec3Length}}
\index{kmVec3Length@{kmVec3Length}!vec3.c@{vec3.c}}
\subsubsection[kmVec3Length]{\setlength{\rightskip}{0pt plus 5cm}kmScalar kmVec3Length (const {\bf kmVec3} $\ast$ {\em pIn})}}
\label{vec3_8c_8e019f2b3549e0dc7b9847868ba548fa}


Returns the length of the vector 

Definition at line 54 of file vec3.c.

References kmSQR(), kmVec3::x, kmVec3::y, and kmVec3::z.

Referenced by kmPlaneNormalize(), and kmVec3Normalize().\hypertarget{vec3_8c_aabb0b5f2fc888052679d401852e63a9}{
\index{vec3.c@{vec3.c}!kmVec3LengthSq@{kmVec3LengthSq}}
\index{kmVec3LengthSq@{kmVec3LengthSq}!vec3.c@{vec3.c}}
\subsubsection[kmVec3LengthSq]{\setlength{\rightskip}{0pt plus 5cm}kmScalar kmVec3LengthSq (const {\bf kmVec3} $\ast$ {\em pIn})}}
\label{vec3_8c_aabb0b5f2fc888052679d401852e63a9}


Returns the square of the length of the vector 

Definition at line 62 of file vec3.c.

References kmSQR(), kmVec3::x, kmVec3::y, and kmVec3::z.

Referenced by kmQuaternionRotationBetweenVec3().\hypertarget{vec3_8c_61a55fc5e4ce21322cb8141653a1d0ff}{
\index{vec3.c@{vec3.c}!kmVec3Normalize@{kmVec3Normalize}}
\index{kmVec3Normalize@{kmVec3Normalize}!vec3.c@{vec3.c}}
\subsubsection[kmVec3Normalize]{\setlength{\rightskip}{0pt plus 5cm}{\bf kmVec3}$\ast$ kmVec3Normalize ({\bf kmVec3} $\ast$ {\em pOut}, \/  const {\bf kmVec3} $\ast$ {\em pIn})}}
\label{vec3_8c_61a55fc5e4ce21322cb8141653a1d0ff}


Returns the vector passed in set to unit length the result is stored in pOut. 

Definition at line 71 of file vec3.c.

References kmScalar, kmVec3Length(), kmVec3::x, kmVec3::y, and kmVec3::z.

Referenced by kmMat4GetForwardVec3(), kmMat4GetRightVec3(), kmMat4GetUpVec3(), kmMat4LookAt(), kmPlaneFromPoints(), kmPlaneNormalize(), kmQuaternionRotationBetweenVec3(), and kmQuaternionToAxisAngle().\hypertarget{vec3_8c_340a910a28c1b7bec6c55fb7072e331f}{
\index{vec3.c@{vec3.c}!kmVec3Scale@{kmVec3Scale}}
\index{kmVec3Scale@{kmVec3Scale}!vec3.c@{vec3.c}}
\subsubsection[kmVec3Scale]{\setlength{\rightskip}{0pt plus 5cm}{\bf kmVec3}$\ast$ kmVec3Scale ({\bf kmVec3} $\ast$ {\em pOut}, \/  const {\bf kmVec3} $\ast$ {\em pIn}, \/  const kmScalar {\em s})}}
\label{vec3_8c_340a910a28c1b7bec6c55fb7072e331f}


Scales a vector to length s. Does not normalize first, you should do that! 

Definition at line 256 of file vec3.c.

References kmVec3::x, kmVec3::y, and kmVec3::z.

Referenced by kmPlaneFromPoints(), and kmQuaternionMultiplyVec3().\hypertarget{vec3_8c_07cb9ab6d64f4a458191f80eb3405324}{
\index{vec3.c@{vec3.c}!kmVec3Subtract@{kmVec3Subtract}}
\index{kmVec3Subtract@{kmVec3Subtract}!vec3.c@{vec3.c}}
\subsubsection[kmVec3Subtract]{\setlength{\rightskip}{0pt plus 5cm}{\bf kmVec3}$\ast$ kmVec3Subtract ({\bf kmVec3} $\ast$ {\em pOut}, \/  const {\bf kmVec3} $\ast$ {\em pV1}, \/  const {\bf kmVec3} $\ast$ {\em pV2})}}
\label{vec3_8c_07cb9ab6d64f4a458191f80eb3405324}


Subtracts 2 vectors and returns the result. The result is stored in pOut. 

Definition at line 140 of file vec3.c.

References kmVec3::x, kmVec3::y, and kmVec3::z.

Referenced by kmMat4LookAt(), kmPlaneFromPoints(), and kmPlaneIntersectLine().\hypertarget{vec3_8c_e0bf920655ecb4211d501af13acaa55b}{
\index{vec3.c@{vec3.c}!kmVec3Transform@{kmVec3Transform}}
\index{kmVec3Transform@{kmVec3Transform}!vec3.c@{vec3.c}}
\subsubsection[kmVec3Transform]{\setlength{\rightskip}{0pt plus 5cm}{\bf kmVec3}$\ast$ kmVec3Transform ({\bf kmVec3} $\ast$ {\em pOut}, \/  const {\bf kmVec3} $\ast$ {\em pV}, \/  const {\bf kmMat4} $\ast$ {\em pM})}}
\label{vec3_8c_e0bf920655ecb4211d501af13acaa55b}


Transforms vector (x, y, z, 1) by a given matrix. The result is stored in pOut. pOut is returned. 

Definition at line 159 of file vec3.c.

References kmMat4::mat, kmVec3::x, kmVec3::y, and kmVec3::z.\hypertarget{vec3_8c_43d9ed3e0b77815f8e21a62d70310d43}{
\index{vec3.c@{vec3.c}!kmVec3TransformCoord@{kmVec3TransformCoord}}
\index{kmVec3TransformCoord@{kmVec3TransformCoord}!vec3.c@{vec3.c}}
\subsubsection[kmVec3TransformCoord]{\setlength{\rightskip}{0pt plus 5cm}{\bf kmVec3}$\ast$ kmVec3TransformCoord ({\bf kmVec3} $\ast$ {\em pOut}, \/  const {\bf kmVec3} $\ast$ {\em pV}, \/  const {\bf kmMat4} $\ast$ {\em pM})}}
\label{vec3_8c_43d9ed3e0b77815f8e21a62d70310d43}


NOT COMPLETE, DO NOT USE! 

Definition at line 217 of file vec3.c.\hypertarget{vec3_8c_7bcd2c7c5d14ca2dcea701e52450b9d9}{
\index{vec3.c@{vec3.c}!kmVec3TransformNormal@{kmVec3TransformNormal}}
\index{kmVec3TransformNormal@{kmVec3TransformNormal}!vec3.c@{vec3.c}}
\subsubsection[kmVec3TransformNormal]{\setlength{\rightskip}{0pt plus 5cm}{\bf kmVec3}$\ast$ kmVec3TransformNormal ({\bf kmVec3} $\ast$ {\em pOut}, \/  const {\bf kmVec3} $\ast$ {\em pV}, \/  const {\bf kmMat4} $\ast$ {\em pM})}}
\label{vec3_8c_7bcd2c7c5d14ca2dcea701e52450b9d9}




Definition at line 230 of file vec3.c.

References kmMat4::mat, kmVec3::x, kmVec3::y, and kmVec3::z.\hypertarget{vec3_8c_b1384be1801802bdd4390481ff018400}{
\index{vec3.c@{vec3.c}!kmVec3Zero@{kmVec3Zero}}
\index{kmVec3Zero@{kmVec3Zero}!vec3.c@{vec3.c}}
\subsubsection[kmVec3Zero]{\setlength{\rightskip}{0pt plus 5cm}{\bf kmVec3}$\ast$ kmVec3Zero ({\bf kmVec3} $\ast$ {\em pOut})}}
\label{vec3_8c_b1384be1801802bdd4390481ff018400}


Sets all the elements of pOut to zero. Returns pOut. 

Definition at line 298 of file vec3.c.

References kmVec3::x, kmVec3::y, and kmVec3::z.